% \iffalse
\let\negmedspace\undefined
\let\negthickspace\undefined
\documentclass[journal,12pt,twocolumn]{IEEEtran}
\usepackage{cite}
\usepackage{amsmath,amssymb,amsfonts,amsthm}
\usepackage{algorithmic}
\usepackage{graphicx}
\usepackage{textcomp}
\usepackage{xcolor}
\usepackage{txfonts}
\usepackage{listings}
\usepackage{enumitem}
\usepackage{mathtools}
\usepackage{gensymb}
\usepackage{comment}
\usepackage[breaklinks=true]{hyperref}
\usepackage{tkz-euclide} 
\usepackage{listings}
\usepackage{gvv}                                        
\def\inputGnumericTable{}                                 
\usepackage[latin1]{inputenc}                                
\usepackage{color}                                            
\usepackage{array}                                            
\usepackage{longtable}                                       
\usepackage{calc}                                             
\usepackage{multirow}                                         
\usepackage{hhline}                                           
\usepackage{ifthen}                                           
\usepackage{lscape}

\newtheorem{theorem}{Theorem}[section]
\newtheorem{problem}{Problem}
\newtheorem{proposition}{Proposition}[section]
\newtheorem{lemma}{Lemma}[section]
\newtheorem{corollary}[theorem]{Corollary}
\newtheorem{example}{Example}[section]
\newtheorem{definition}[problem]{Definition}
\newcommand{\BEQA}{\begin{eqnarray}}
\newcommand{\EEQA}{\end{eqnarray}}
\newcommand{\define}{\stackrel{\triangle}{=}}
\theoremstyle{remark}
\newtheorem{rem}{Remark}
\begin{document}

\bibliographystyle{IEEEtran}
\vspace{3cm}

\Large\title{NCERT Question 11.9.3.15}
\large\author{EE23BTECH11032 - Kaustubh Parag Khachane $^{*}$% <-this % stops a space
}
\maketitle
\newpage
\bigskip

\renewcommand{\thefigure}{\theenumi}
\renewcommand{\thetable}{\theenumi}
\large\textbf{Question 11.9.3.15} : Given a GP with a = 729 and \(7^{th}\) term 64, determine $S_{7}$

\vspace{4mm} 

\large\textbf{Solution} :\normalsize

\vspace{4mm} 

The \(n^{th}\) term of a GP is given by a*\(r^{n-1}\) where a is the first term of the GP and r is the common difference.

\vspace{4mm}

The first term of the GP (a) = 729\\
The \(7^{th}\) term is given by a*\(r^{6}\)\\
The seventh term is given as 64

\begin{align}
&\therefore  64 = 729*(r^{6})\notag\\
&\implies \left(\frac{64}{729}\right) = r^6\notag\\
&\implies \frac{2}{3} = r \notag
\end{align}

The sum of n terms of a GP is given by :
\begin{align}
&S_{n} = \Large\frac{a*(r^n - 1)}{r-1} \notag
\end{align}
\begin{equation} \label{eq1}
\begin{split}
\therefore S_{7} & = \frac{729*\left(\left(\frac{2}{3}\right)^7 -1\right)}{\frac{2}{3} - 1} \notag\\
 & = \frac{729*\left(1 - \left(\frac{128}{2187}\right)\right)}{1 - \frac{2}{3}} \notag\\
  & = \frac{729*\left(\frac{2187 - 128}{2187}\right)}{\frac{1}{3}}\notag\\
  & = \frac{729*\left(\frac{2059}{2187}\right)}{\frac{1}{3}}\notag\\
  & = \frac{729*3*2059}{2187}\notag\\
  & = 2059\notag
\end{split}
\end{equation}
\large\textbf{Answer}: \normalsize The sum of the first 7 terms of the GP ($S_{7}$) is 2059
\end{document}
