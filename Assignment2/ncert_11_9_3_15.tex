% \iffalse
\let\negmedspace\undefined
\let\negthickspace\undefined
\documentclass[journal,12pt,twocolumn]{IEEEtran}
\usepackage{cite}
\usepackage{amsmath,amssymb,amsfonts,amsthm}
\usepackage{algorithmic}
\usepackage{graphicx}
\usepackage{textcomp}
\usepackage{xcolor}
\usepackage{txfonts}
\usepackage{listings}
\usepackage{enumitem}
\usepackage{mathtools}
\usepackage{gensymb}
\usepackage{comment}
\usepackage[breaklinks=true]{hyperref}
\usepackage{tkz-euclide} 
\usepackage{listings}
\usepackage{gvv}                                        
\def\inputGnumericTable{}                                 
\usepackage[latin1]{inputenc}                                
\usepackage{color}                                            
\usepackage{array}                                            
\usepackage{longtable}                                       
\usepackage{calc}                                             
\usepackage{multirow}                                         
\usepackage{hhline}                                           
\usepackage{ifthen}                                           
\usepackage{lscape}

\newtheorem{theorem}{Theorem}[section]
\newtheorem{problem}{Problem}
\newtheorem{proposition}{Proposition}[section]
\newtheorem{lemma}{Lemma}[section]
\newtheorem{corollary}[theorem]{Corollary}
\newtheorem{example}{Example}[section]
\newtheorem{definition}[problem]{Definition}
\newcommand{\BEQA}{\begin{eqnarray}}
\newcommand{\EEQA}{\end{eqnarray}}
\newcommand{\define}{\stackrel{\triangle}{=}}
\theoremstyle{remark}
\newtheorem{rem}{Remark}
\begin{document}

\bibliographystyle{IEEEtran}
\vspace{3cm}

\Large\title{NCERT Question 11.9.3.15}
\large\author{EE23BTECH11032 - Kaustubh Parag Khachane $^{*}$% <-this % stops a space
}
\maketitle
\newpage
\bigskip

\renewcommand{\thefigure}{\theenumi}
\renewcommand{\thetable}{\theenumi}
\large\textbf{Question 11.9.3.15} : Given a GP with $x_0$ = 729 and \(7^{th}\) term 64, determine $S_{7}$

\solution

\begin{table}[!ht] 
\centering
\setlength{\extrarowheight}{8pt}
\begin{tabular}{|l|l|l|}
    \hline
    \textbf{Parameter} & \textbf{Description} & \textbf{Value} \\
    \hline
     m & Mass of object & 10 Kg \\\hline
     $\mu$ & Frictional coefficient \brak{static} & 0.25\\\hline
     x\brak{t} & Displacement of block &  \\\hline
     $x\brak{0}$ & Initial displacement & 0 \brak{assumed} \\\hline
     g & Gravitational acceleration & 10 $m/s^2$ \\\hline
     $F_s$ & Spring force &  \\\hline
     f & frictional force &  $\mu$ N \\\hline
     N & Normal Force & mg $cos\brak{\theta}$ \\\hline
    \end{tabular}
  \vspace{4mm}
 \caption{Parameter Table}
 \label{tab:table0_xe80}
\end{table}

\begin{align}
    S_k = x\brak{0}\frac{r^k -1}{r-1} \label{eq:eq1}
\end{align}

from \tabref{tab:table0} :
\begin{align}
    &x\brak{6} = x\brak{0}r^6\\
    &\implies 64 = 729r^6\\
    &\therefore r = \frac{2}{3}\label{eq:eq2}
\end{align}
    using equation \eqref{eq:eq1} and equation \eqref{eq:eq2}
\begin{align}
    S_7 &= 729\frac{\brak{\frac{2}{3}}^6 - 1}{\frac{2}{3} - 1}\\
  & = \frac{729\left(\frac{2187 - 128}{2187}\right)}{\frac{1}{3}}\\
  & = 2059\notag
\end{align}

\end{document}
