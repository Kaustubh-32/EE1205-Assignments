% \iffalse
\let\negmedspace\undefined
\let\negthickspace\undefined
\documentclass[journal,12pt,twocolumn]{IEEEtran}
\usepackage{cite}
\usepackage{amsmath,amssymb,amsfonts,amsthm}
\usepackage{algorithmic}
\usepackage{graphicx}
\usepackage{textcomp}
\usepackage{xcolor}
\usepackage{txfonts}
\usepackage{listings}
\usepackage{enumitem}
\usepackage{mathtools}
\usepackage{gensymb}
\usepackage{comment}
\usepackage[breaklinks=true]{hyperref}
\usepackage{tkz-euclide} 
\usepackage{listings}
\usepackage{gvv}                                        
\def\inputGnumericTable{}                                 
\usepackage[latin1]{inputenc}                                
\usepackage{color}                                            
\usepackage{array}                                            
\usepackage{longtable}                                       
\usepackage{calc}                                             
\usepackage{multirow}                                         
\usepackage{hhline}                                           
\usepackage{ifthen}                                           
\usepackage{lscape}

\newtheorem{theorem}{Theorem}[section]
\newtheorem{problem}{Problem}
\newtheorem{proposition}{Proposition}[section]
\newtheorem{lemma}{Lemma}[section]
\newtheorem{corollary}[theorem]{Corollary}
\newtheorem{example}{Example}[section]
\newtheorem{definition}[problem]{Definition}
\newcommand{\BEQA}{\begin{eqnarray}}
\newcommand{\EEQA}{\end{eqnarray}}
\newcommand{\define}{\stackrel{\triangle}{=}}
\theoremstyle{remark}
\newtheorem{rem}{Remark}

\graphicspath{{./figs/}}

\begin{document}

\bibliographystyle{IEEEtran}
\vspace{3cm}

\Large\title{NCERT Question 11.9.3.15}
\large\author{EE23BTECH11032 - Kaustubh Parag Khachane $^{*}$% <-this % stops a space
}
\maketitle
\newpage
\bigskip

\renewcommand{\thefigure}{\theenumi}
\renewcommand{\thetable}{\theenumi}
\large\textbf{Question 11.9.3.15} : Given a GP with x\brak{0} = 729 and \(7^{th}\) term 64, determine s\brak{6}

\solution

\begin{table}[ht] \label{table1}
\centering
\setlength{\extrarowheight}{10pt}
\begin{tabular}{|c|c|c|} 
 \hline
  \textbf{Parameter} & \textbf{Used to denote } & \textbf{Values} \\ 
 \hline
 $x_{1}$\brak{n} & $n^{th}$ term of $1^{st}$ series & 7 + 6n  \\
 \hline
 $x_{2}$\brak{n} & $n^{th}$ term of $2^{nd}$ series & 18 -2$\frac{1}{2}$n \\
 \hline
$x_{1}$\brak{0} & First term of $1^{st}$ AP & 7\\ 
 \hline
 $x_{2}$\brak{0} & First term of $2^{nd}$ AP & 18\\ 
 \hline
  $d_{1}$ & Common difference of $1^{st}$ AP & 6 \\
 \hline
 $d_{2}$& Common difference of $2^{nd}$ & -2$\frac{1}{2}$ \\[5pt]
 \hline
 
 $X_{i}$\brak{z} & z transform of $x_{i}$\brak{n}, i= 1 or 2 & $\sum_{n=-\infty}^{\infty}x\brak{n}z^{-n}$ \\
 \hline
 u\brak{n} & unit step function & $\begin{cases}   1 \text { for } n \geq 0\\
            0 \text{ for } n < 0  \end{cases}$ \\
 \hline
  U\brak{z} & z transform of u\brak{n} & $\sum_{n=0}^{\infty}z^{-n} =  \frac{1}{1-z^{-1}}$\\
  \hline
 ROC & Region of Convergence & To find \\
 \hline
\end{tabular}
 \vspace{4mm}
 \caption{Parameter Table}
 \label{table0}
\end{table}

\begin{align}
   & X\brak{z} = \frac{x\brak{0}}{1 - rz^{-1}} \label{eq:eq3}\\
   & \text{ROC is $\abs{z} > \abs{r}$ as it is a GP.}   
\end{align}
 Sum to n terms of GP can be given as :
\begin{align}
    s\brak{n} &= x\brak{n}*u\brak{n}\\
    \implies  S\brak{z} &= X\brak{z}U\brak{z}\label{eq:eq4}
\end{align}
from \tabref{tab:table0} :
\begin{align}
    &x\brak{6} = x\brak{0}r^6\\
    \implies &64 = 729r^6\\
    &\therefore r = \frac{2}{3}\label{eq:eq2}
\end{align}
%    using equation \eqref{eq:eq1} and equation \eqref{eq:eq2}
%\begin{align}
% s\brak{6} &= 729\frac{\brak{\frac{2}{3}}^6 - 1}{\frac{2}{3} - 1}\\
% & = \frac{729\left(\frac{2187 - 128}{2187}\right)}{\frac{1}{3}}\\
%\implies s\brak{6} &= 2059
%\end{align}
using \tabref{tab:table0} and equation \eqref{eq:eq3}
\begin{align}
    X\brak{z} &= \frac{729}{1 - \frac{2}{3}z^{-1}}\label{eq:eq6}
\end{align}
using \tabref{tab:table0}, equation \eqref{eq:eq4} and equation \eqref{eq:eq6}
\begin{align}
    S\brak{z} &= \frac{729}{\brak{1 - \frac{2}{3}z^{-1}}\brak{1-z^{-1}}}\\
    &= 2187\brak{\frac{1}{1-z^{-1}} - \frac{\frac{2}{3}}{1 - \frac{2}{3}z^{-1}}} \label{eq:eq5}
\end{align}
Using contour integration for inverse z transform,
\begin{align}
    s\brak{6} &= \frac{1}{2\pi j}\int S\brak{z} z^{5} dz\\
    &= \frac{1}{2\pi j}\brak{\int\frac{2187z^{6}}{z-1}dz + \int\frac{1458z^{6}}{z - \frac{2}{3}} dz} \label{eq:eq8}
\end{align}
Solution of each of these integrals can be given by :
\begin{align}
    I =\frac{1}{\brak {m-1}!}\lim\limits_{z\to a}\frac{d^{m-1}}{dz^{m-1}}\brak {{(z-a)}^{m}f\brak z} \label{eq:eq7}
\end{align}
using equations \eqref{eq:eq8} and \eqref{eq:eq7}:
\begin{align}
    \frac{1}{2\pi j}\brak{\int\frac{2187z^{6}}{z - 1}dz} &= 2187\label{eq:eq9}\\
     \frac{1}{2\pi j}\brak{\int\frac{1458z^{6}}{z - \frac{2}{3}}dz}  &= 128\label{eq:eq10}
     \end{align}
using equations \eqref{eq:eq8}, \eqref{eq:eq9}, \eqref{eq:eq10}:
\begin{align}
s\brak{6} &= 2187 - 128\\
&= 2059
\end{align}
\begin{figure}[!ht]
\centering
\begin{center}
\includegraphics[width=\columnwidth]{Figure_1}
\caption{Plot of $s\brak{n}$}
\end{center}
\end{figure}

\begin{figure}[!ht]
\centering
\begin{center}
\includegraphics[width=\columnwidth]{Figure_2}
\caption{Plot of $x\brak{n}$}
\end{center}
\end{figure}
\end{document}
