% \iffalse
\let\negmedspace\undefined
\let\negthickspace\undefined
\documentclass[journal,12pt,twocolumn]{IEEEtran}
\usepackage{cite}
\usepackage{amsmath,amssymb,amsfonts,amsthm}
\usepackage{algorithmic}
\usepackage{graphicx}
\usepackage{textcomp}
\usepackage{xcolor}
\usepackage{txfonts}
\usepackage{listings}
\usepackage{enumitem}
\usepackage{mathtools}
\usepackage{gensymb}
\usepackage{comment}
\usepackage[breaklinks=true]{hyperref}
\usepackage{tkz-euclide} 
\usepackage{listings}
\usepackage{gvv}                                        
\def\inputGnumericTable{}                                 
\usepackage[latin1]{inputenc}                                
\usepackage{color}                                            
\usepackage{array}                                            
\usepackage{longtable}                                       
\usepackage{calc}                                             
\usepackage{multirow}                                         
\usepackage{hhline}                                           
\usepackage{ifthen}                                           
\usepackage{lscape}
\usepackage{placeins}
\usepackage{xparse}


\newtheorem{theorem}{Theorem}[section]
\newtheorem{problem}{Problem}
\newtheorem{proposition}{Proposition}[section]
\newtheorem{lemma}{Lemma}[section]
\newtheorem{corollary}[theorem]{Corollary}
\newtheorem{example}{Example}[section]
\newtheorem{definition}[problem]{Definition}
\newcommand{\BEQA}{\begin{eqnarray}}
\newcommand{\EEQA}{\end{eqnarray}}
\newcommand{\define}{\stackrel{\triangle}{=}}
\theoremstyle{remark}
\newtheorem{rem}{Remark}

\graphicspath{ {./figs/} } 

\begin{document}

\bibliographystyle{IEEEtran}
\vspace{3cm}

\Large\title{GATE 2022 EE 28}
\large\author{EE23BTECH11032 - Kaustubh Parag Khachane $^{*}$% <-this % stops a space
}
\maketitle
\newpage
\bigskip

\renewcommand{\thefigure}{\theenumi}
\renewcommand{\thetable}{\theenumi}
\large\textbf{Question GATE 22 EE 28} :\\
The network shown below has a resonant frequency of 150 kHz and a bandwidth of
600 Hz. The $Q$-factor of the network is \rule{1cm}{0.15mm}
\begin{figure}[!ht]
\centering
\begin{center}
\includegraphics[width=\columnwidth]{question}
\end{center}
%\caption{Diagram for GATE ME Question 30}
\end{figure}
\hfill(GATE EE 2022)\\
\solution\\
\begin{table}[ht] \label{table1}
\centering
\setlength{\extrarowheight}{10pt}
\begin{tabular}{|c|c|c|} 
 \hline
  \textbf{Parameter} & \textbf{Used to denote } & \textbf{Values} \\ 
 \hline
 $x_{1}$\brak{n} & $n^{th}$ term of $1^{st}$ series & 7 + 6n  \\
 \hline
 $x_{2}$\brak{n} & $n^{th}$ term of $2^{nd}$ series & 18 -2$\frac{1}{2}$n \\
 \hline
$x_{1}$\brak{0} & First term of $1^{st}$ AP & 7\\ 
 \hline
 $x_{2}$\brak{0} & First term of $2^{nd}$ AP & 18\\ 
 \hline
  $d_{1}$ & Common difference of $1^{st}$ AP & 6 \\
 \hline
 $d_{2}$& Common difference of $2^{nd}$ & -2$\frac{1}{2}$ \\[5pt]
 \hline
 
 $X_{i}$\brak{z} & z transform of $x_{i}$\brak{n}, i= 1 or 2 & $\sum_{n=-\infty}^{\infty}x\brak{n}z^{-n}$ \\
 \hline
 u\brak{n} & unit step function & $\begin{cases}   1 \text { for } n \geq 0\\
            0 \text{ for } n < 0  \end{cases}$ \\
 \hline
  U\brak{z} & z transform of u\brak{n} & $\sum_{n=0}^{\infty}z^{-n} =  \frac{1}{1-z^{-1}}$\\
  \hline
 ROC & Region of Convergence & To find \\
 \hline
\end{tabular}
 \vspace{4mm}
 \caption{Parameter Table}
 \label{table0}
\end{table}

In the state of resonance ,
\begin{align}
    \omega_0 &= \frac{1}{\sqrt{LC}}\label{eq:eq1}\\
    f_0 &= \frac{1}{2\pi \sqrt{LC}}
\end{align}
Bandwidth is range of frequencies where power is $\geq$ maximum power.\\
Hence it is the range of frequencies between the two points where power is half.\\
\begin{align}
    \text{Power is half }\implies I = \frac{I_0}{\sqrt{2}}
\end{align}
Current at any point is given by ,
\begin{align}
    \frac{V}{\sqrt{R^2 + \brak{\omega L - \frac{1}{\omega C}}^2}} = \frac{V}{R\sqrt{2}}
\end{align}
Solving the above equation, we will get two solution $\omega_1$ and $\omega_2$ which will satisfy
\begin{align}
    \brak{\omega L - \frac{1}{\omega C}} &= \pm R\\
    \implies \omega_2 - \omega_1 = \frac{R}{L}\\
    \therefore \text{Bandwidth } = \frac{R}{L} \label{eq:eq2}
\end{align}
\begin{align}
    Q \text{ factor } &= 2\pi\frac{\text{Peak energy stored}}{\text{Energy disspated in one cycle}} \\
    &= \frac{\frac{1}{2}LI^2}{\frac{1}{2} I^2R\frac{1}{f_0}}\\
    &= \frac{\omega_0}{\frac{R}{L}}\\
\end{align}
Using equations \eqref{eq:eq1} and \eqref{eq:eq2},
\begin{align}
    Q &= \frac{150000}{600}\\
    &= 250
\end{align}
\end{document}
