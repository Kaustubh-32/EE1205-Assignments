% \iffalse
\let\negmedspace\undefined
\let\negthickspace\undefined
\documentclass[journal,12pt,twocolumn]{IEEEtran}
\usepackage{cite}
\usepackage{amsmath,amssymb,amsfonts,amsthm}
\usepackage{algorithmic}
\usepackage{graphicx}
\usepackage{textcomp}
\usepackage{xcolor}
\usepackage{txfonts}
\usepackage{listings}
\usepackage{enumitem}
\usepackage{mathtools}
\usepackage{gensymb}
\usepackage{comment}
\usepackage[breaklinks=true]{hyperref}
\usepackage{tkz-euclide} 
\usepackage{listings}
\usepackage{gvv}                                        
\def\inputGnumericTable{}                                 
\usepackage[latin1]{inputenc}                                
\usepackage{color}                                            
\usepackage{array}                                            
\usepackage{longtable}                                       
\usepackage{calc}                                             
\usepackage{multirow}                                         
\usepackage{hhline}                                           
\usepackage{ifthen}                                           
\usepackage{lscape}
\usepackage{placeins}
\usepackage{xparse}


\newtheorem{theorem}{Theorem}[section]
\newtheorem{problem}{Problem}
\newtheorem{proposition}{Proposition}[section]
\newtheorem{lemma}{Lemma}[section]
\newtheorem{corollary}[theorem]{Corollary}
\newtheorem{example}{Example}[section]
\newtheorem{definition}[problem]{Definition}
\newcommand{\BEQA}{\begin{eqnarray}}
\newcommand{\EEQA}{\end{eqnarray}}
\newcommand{\define}{\stackrel{\triangle}{=}}
\theoremstyle{remark}
\newtheorem{rem}{Remark}

\graphicspath{ {./figs/} } 

\begin{document}

\bibliographystyle{IEEEtran}
\vspace{3cm}

\Large\title{NCERT Question 10.5.2.5}
\large\author{EE23BTECH11032 - Kaustubh Parag Khachane $^{*}$% <-this % stops a space
}
\maketitle
\newpage
\bigskip

\renewcommand{\thefigure}{\theenumi}
\renewcommand{\thetable}{\theenumi}
\large\textbf{Question 10.5.2.5} : \normalsize Find the number of terms in each of the following APs. Then express each term as x(n) and find the z transform and its ROC: 

\brak{i} 7, 13, 19, ... 205

\brak{ii} 18, 15$\frac{1}{2}$, 13, ... -47


\large\textbf{Solution} :\normalsize

\begin{table}[ht] \label{table1}
\centering
\setlength{\extrarowheight}{10pt}
\begin{tabular}{|c|c|c|} 
 \hline
  \textbf{Parameter} & \textbf{Used to denote } & \textbf{Values} \\ 
 \hline
 $x_{1}$\brak{n} & $n^{th}$ term of $1^{st}$ series & 7 + 6n  \\
 \hline
 $x_{2}$\brak{n} & $n^{th}$ term of $2^{nd}$ series & 18 -2$\frac{1}{2}$n \\
 \hline
$x_{1}$\brak{0} & First term of $1^{st}$ AP & 7\\ 
 \hline
 $x_{2}$\brak{0} & First term of $2^{nd}$ AP & 18\\ 
 \hline
  $d_{1}$ & Common difference of $1^{st}$ AP & 6 \\
 \hline
 $d_{2}$& Common difference of $2^{nd}$ & -2$\frac{1}{2}$ \\[5pt]
 \hline
 
 $X_{i}$\brak{z} & z transform of $x_{i}$\brak{n}, i= 1 or 2 & $\sum_{n=-\infty}^{\infty}x\brak{n}z^{-n}$ \\
 \hline
 u\brak{n} & unit step function & $\begin{cases}   1 \text { for } n \geq 0\\
            0 \text{ for } n < 0  \end{cases}$ \\
 \hline
  U\brak{z} & z transform of u\brak{n} & $\sum_{n=0}^{\infty}z^{-n} =  \frac{1}{1-z^{-1}}$\\
  \hline
 ROC & Region of Convergence & To find \\
 \hline
\end{tabular}
 \vspace{4mm}
 \caption{Parameter Table}
 \label{table0}
\end{table}

The number of terms in the AP x(n) is given by : 
\begin{align}  \label{eq:eq12}
    \frac{x\brak{n} - x\brak{0}}{d}
\end{align}

\begin{table}[ht] \label{table2}
\centering
\setlength{\extrarowheight}{10pt}
\begin{tabular}{|c|l|l|} 
 \hline
  \textbf{Paramter} & \textbf{AP} & \textbf{Value} \\ 
 \hline
n & 7, 13, 19, ... 205 & $\frac{205 - 7}{6} = 33$\\
 \hline
n & 18, 15$\frac{1}{2}, 13, ... -47$ & $\frac{-47 - 18 }{-2\frac{1}{2}} = 26$\\
 \hline
\end{tabular}
 \vspace{4mm}
 \caption{Answer Table : number of elements}
 \label{table1}
\end{table}


Finding the z transform and ROC : \\
\begin{align}
\text{For an AP } x\brak{n} &= \sbrak{x\brak{0} + nd}u\brak{n}\\
&=x\brak{0}u\brak{n} + dnu\brak{n}\label{eq:eq2}\\
   z\{u\brak{n}\} &=  U\brak{z} = \frac{1}{1 - z^{-1}} , |z| > 1\label{eq:eq4}\\
   z\{nu\brak{n}\} &= -z\frac{dU\brak{z}}{dz} = \frac{z^{-1}}{\brak{1-z^{-1}}^2},|z| > 1 \label{eq:eq5}
\end{align}
ROC is given by values of z for which :
\begin{align}
    \lvert X\brak{z}\rvert = \sum_{n = -\infty}^{\infty}\lvert x\brak{n}z^{-n}\rvert < \infty
\end{align}
Using equations \eqref{eq:eq4} and \eqref{eq:eq5} in equation \eqref{eq:eq2} :
\begin{align}\label{eq:eq3}
    z\{x\brak{n}\} = X(z) = \frac{x\brak{0}}{1 - z^{-1}} + d\frac{z^{-1}}{\brak{1-z^{-1}}^2}
\end{align}
\textbf{\brak{i}}

\begin{align}
x_{1}\brak{n} &= \brak{7 + \brak{n}6}u\brak{n}\\
     x_{1}\brak{n} & = \begin{cases} 
        0 & \text{for } n < 0 \\
        7 + \brak{n}6 & \text{for } n \geq 0
    \end{cases}
\end{align}
Using the values in \tabref{table0} and equation \eqref{eq:eq3} :
\begin{align}
X_{1}\brak{z} &= \frac{7}{1 - z^{-1}} + \frac{6z^{-1}}{\brak{1 - z^{-1}}^2}\\
 &= \frac{7 - z^{-1}}{\brak{1-z^{-1}}^2}
 \end{align}
 The ROC is $|z|>1$ as it is an AP.
 
\textbf{\brak{ii}}

\begin{align}
x_{2}\brak{n} &= \brak{18 + n\brak{-2\frac{1}{2}}}u\brak{n}\\
     x_{2}\brak{n} & = \begin{cases}
        0 & \text{for } n < 0 \\
        18 + n\brak{-2\frac{1}{2}} & \text{for } n \geq 0
    \end{cases}
 \end{align}
Using the values in \tabref{table0} and equation \eqref{eq:eq3} :
\begin{align} 
 X\brak{z} &=  \frac{18}{1 - z^{-1}} - \brak{2\frac{1}{2}}\frac{z^{-1}}{\brak{1-z^{-1}}^2}\\
&= \frac{18 - \brak{20\frac{1}{2}}z^{-1}}{\brak{1 - z^{-1}}^2}
\end{align}

The ROC is $|z|>1$ as it is an AP.

\newpage
The graphs for x\brak{n} : \\
\large\textbf{\brak{i}} \normalsize The graph of $x_{1}$\brak{n} is :
\begin{figure}[ht]
    \begin{center}
    \includegraphics[width = 8cm]{Figure_1}\\
    Fig. 0. Plot of x\brak{n} \\
    \end{center}
\end{figure}

\large\textbf{\brak{ii}} \normalsize The graph of $x_{2}$\brak{n} is :
\begin{figure}[!ht]
    \begin{center}
    \includegraphics[width = 8cm]{Figure_2}\\
    Fig. 1. Plot of x\brak{n} \\
    \end{center}
\end{figure}

\end{document}
