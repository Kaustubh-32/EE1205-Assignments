% \iffalse
\let\negmedspace\undefined
\let\negthickspace\undefined
\documentclass[journal,12pt,twocolumn]{IEEEtran}
\usepackage{cite}
\usepackage{amsmath,amssymb,amsfonts,amsthm}
\usepackage{algorithmic}
\usepackage{graphicx}
\usepackage{textcomp}
\usepackage{xcolor}
\usepackage{txfonts}
\usepackage{listings}
\usepackage{enumitem}
\usepackage{mathtools}
\usepackage{gensymb}
\usepackage{comment}
\usepackage[breaklinks=true]{hyperref}
\usepackage{tkz-euclide} 
\usepackage{listings}
\usepackage{gvv}                                        
\def\inputGnumericTable{}                                 
\usepackage[latin1]{inputenc}                                
\usepackage{color}                                            
\usepackage{array}                                            
\usepackage{longtable}                                       
\usepackage{calc}                                             
\usepackage{multirow}                                         
\usepackage{hhline}                                           
\usepackage{ifthen}                                           
\usepackage{lscape}

\newtheorem{theorem}{Theorem}[section]
\newtheorem{problem}{Problem}
\newtheorem{proposition}{Proposition}[section]
\newtheorem{lemma}{Lemma}[section]
\newtheorem{corollary}[theorem]{Corollary}
\newtheorem{example}{Example}[section]
\newtheorem{definition}[problem]{Definition}
\newcommand{\BEQA}{\begin{eqnarray}}
\newcommand{\EEQA}{\end{eqnarray}}
\newcommand{\define}{\stackrel{\triangle}{=}}
\theoremstyle{remark}
\newtheorem{rem}{Remark}
\begin{document}

\bibliographystyle{IEEEtran}
\vspace{3cm}

\Large\title{NCERT Question 10.5.2.5}
\large\author{EE23BTECH11032 - Kaustubh Parag Khachane $^{*}$% <-this % stops a space
}
\maketitle
\newpage
\bigskip

\renewcommand{\thefigure}{\theenumi}
\renewcommand{\thetable}{\theenumi}
\large\textbf{Question 10.5.2.5} : \normalsize Find the number of terms in each of the following APs : 

(i) 7, 13, 19, ... 205

(ii) 18, 15\(\frac{1}{2}\), 13, ... -47

\vspace{4mm} 

\large\textbf{Solution} :\normalsize

\vspace{4mm}

\textbf{(i)} The \(n^{th}\) term of the Arithmetic progression is given as a + (n-1)*d where a is the first term and d is the common difference.

The common difference of the AP is given by the difference between successive terms.

\vspace{4mm}

Common difference (d) = 13 - 7 = 6

First term (a) = 7

\vspace{4mm}

If 205 is the \(n{th}\) term of the series, we have :
\begin{align}
&205 = 7 + (n-1)*6 \notag\\ 
\implies&  198 = (n-1)*6 \notag\\
\implies&  33 = n-1\notag\\
\implies& n = 34\notag
\end{align}

\large\textbf{Answer} : \normalsize There are 34 elements in the series.

\vspace{4mm}

\textbf{(ii)} The \(n^{th}\) term of the Arithmetic progression is given as a + (n-1)*d where a is the first term and d is the common difference.

The common difference of the AP is given by the difference between successive terms.

\vspace{4mm}

Common difference (d) = 15\(\frac{1}{2}\) - 18 = -2\(\frac{1}{2}\)

First term (a) = 18

\vspace{4mm}

If -47 is the \(n{th}\) term of the series, we have :
\begin{align}
&-47 = 18 + (n-1)*\left(-2\left(\frac{1}{2}\right)\right) \notag\\ 
\implies& -65 = (n-1)*\left(-2\left(\frac{1}{2}\right)\right) \notag\\
\implies& 26 = n-1\notag\\
\implies& n = 27\notag
\end{align} 

\large\textbf{Answer} : \normalsize There are 27 elements in the series.
\end{document}
