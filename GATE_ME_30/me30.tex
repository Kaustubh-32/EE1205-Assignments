% \iffalse
\let\negmedspace\undefined
\let\negthickspace\undefined
\documentclass[journal,12pt,twocolumn]{IEEEtran}
\usepackage{cite}
\usepackage{amsmath,amssymb,amsfonts,amsthm}
\usepackage{algorithmic}
\usepackage{graphicx}
\usepackage{textcomp}
\usepackage{xcolor}
\usepackage{txfonts}
\usepackage{listings}
\usepackage{enumitem}
\usepackage{mathtools}
\usepackage{gensymb}
\usepackage{comment}
\usepackage[breaklinks=true]{hyperref}
\usepackage{tkz-euclide} 
\usepackage{listings}
\usepackage{gvv}                                        
\def\inputGnumericTable{}                                 
\usepackage[latin1]{inputenc}                                
\usepackage{color}                                            
\usepackage{array}                                            
\usepackage{longtable}                                       
\usepackage{calc}                                             
\usepackage{multirow}                                         
\usepackage{hhline}                                           
\usepackage{ifthen}                                           
\usepackage{lscape}
\usepackage{placeins}
\usepackage{xparse}


\newtheorem{theorem}{Theorem}[section]
\newtheorem{problem}{Problem}
\newtheorem{proposition}{Proposition}[section]
\newtheorem{lemma}{Lemma}[section]
\newtheorem{corollary}[theorem]{Corollary}
\newtheorem{example}{Example}[section]
\newtheorem{definition}[problem]{Definition}
\newcommand{\BEQA}{\begin{eqnarray}}
\newcommand{\EEQA}{\end{eqnarray}}
\newcommand{\define}{\stackrel{\triangle}{=}}
\theoremstyle{remark}
\newtheorem{rem}{Remark}

\graphicspath{ {./figs/} } 

\begin{document}

\bibliographystyle{IEEEtran}
\vspace{3cm}

\Large\title{GATE ME 30}
\large\author{EE23BTECH11032 - Kaustubh Parag Khachane $^{*}$% <-this % stops a space
}
\maketitle
\newpage
\bigskip

\renewcommand{\thefigure}{\theenumi}
\renewcommand{\thetable}{\theenumi}
\large\textbf{Question GATE ME 30} :\\
The figure shows a block of mass m = 20 kg attached to a pair of identical linear springs, each having a spring constant k = 1000 N/m. The block oscillates on a frictionless horizontal surface. Assuming free vibrations, the time taken by the block to complete ten oscillations is \rule{1cm}{0.15mm} seconds . (Rounded off to two decimal places) Take $\pi$ = 3.14. \\ \hfill(GATE ME 2023)

\begin{figure}[!ht]
\centering
\begin{center}
\includegraphics[width=\columnwidth]{questiondiagram}
\end{center}
%\caption{Diagram for GATE ME Question 30}
\end{figure}

\solution\\
\begin{table}[ht] \label{table1}
\centering
\setlength{\extrarowheight}{10pt}
\begin{tabular}{|c|c|c|} 
 \hline
  \textbf{Parameter} & \textbf{Used to denote } & \textbf{Values} \\ 
 \hline
 $x_{1}$\brak{n} & $n^{th}$ term of $1^{st}$ series & 7 + 6n  \\
 \hline
 $x_{2}$\brak{n} & $n^{th}$ term of $2^{nd}$ series & 18 -2$\frac{1}{2}$n \\
 \hline
$x_{1}$\brak{0} & First term of $1^{st}$ AP & 7\\ 
 \hline
 $x_{2}$\brak{0} & First term of $2^{nd}$ AP & 18\\ 
 \hline
  $d_{1}$ & Common difference of $1^{st}$ AP & 6 \\
 \hline
 $d_{2}$& Common difference of $2^{nd}$ & -2$\frac{1}{2}$ \\[5pt]
 \hline
 
 $X_{i}$\brak{z} & z transform of $x_{i}$\brak{n}, i= 1 or 2 & $\sum_{n=-\infty}^{\infty}x\brak{n}z^{-n}$ \\
 \hline
 u\brak{n} & unit step function & $\begin{cases}   1 \text { for } n \geq 0\\
            0 \text{ for } n < 0  \end{cases}$ \\
 \hline
  U\brak{z} & z transform of u\brak{n} & $\sum_{n=0}^{\infty}z^{-n} =  \frac{1}{1-z^{-1}}$\\
  \hline
 ROC & Region of Convergence & To find \\
 \hline
\end{tabular}
 \vspace{4mm}
 \caption{Parameter Table}
 \label{table0}
\end{table}

\begin{align}
    F &= ma \\
    F &= -kx \\
    \implies ma + kx &= 0\\
    \therefore m\frac{d^2x}{dt^2} + kx &= 0\label{eq:eq1}
\end{align}
The Laplace transform of the terms is ,
\begin{align}
    \frac{d^2x}{dt^2} & \system{\mathcal{L}} s^2 X\brak{s} - sx\brak{0} - \dot{x}\brak{0}\label{eq:eq2}\\
    x & \system{\mathcal{L}} X\brak{s}\label{eq:eq3}
\end{align}
Using equation \eqref{eq:eq2} and \eqref{eq:eq3} in equation \eqref{eq:eq1},
\begin{align}
    &m\brak{s^2 X\brak{s} - sx\brak{0} - \dot{x}\brak{0}} + kX\brak{s} = 0\\
    &ms^2X\brak{s} -msA + m\brak{0} + kX\brak{s} = 0\\
    &X\brak{s} = \frac{msA}{ms^2 + k} \\
    \implies &X\brak{s} = \frac{sA}{s^2 + \frac{k}{m}} \label{eq:eq4}
\end{align}
The inverse Laplace transform of such terms is given by,
\begin{align}
    \frac{s}{s^2 + a^2} \system{\mathcal{L^{ -}}} cos\brak{at}u\brak{t}
\end{align}
$\therefore$ the inverse Laplace of \eqref{eq:eq4} is,
\begin{align}
    x\brak{t} = Acos\brak{\sqrt{\frac{k}{m}t}} \label{eq:eq5}
\end{align}
From equation \eqref{eq:eq5} and \tabref{tab:table0} ,the time to complete one oscillation is,
\begin{align}
    T_n &= \frac{2\pi}{\sqrt{\frac{k}{m}}}\\
    &= \frac{\pi}{5}\label{eq:eq6}
\end{align}
$\therefore$ the time required for 10 oscillations is ,
\begin{align}
    10T_n &= 2\pi\\
    &= 6.28 s
\end{align}
\begin{figure}[!ht]
\centering
\begin{center}
\includegraphics[width=\columnwidth]{Figure_1}
\end{center}
\caption{Plot of $x\brak{t}$}
\end{figure}
\end{document}
